\documentclass[english,onecolumn]{IEEEtran}

\usepackage[T1]{fontenc}
\usepackage[latin9]{luainputenc}
\usepackage[letterpaper]{geometry}
\geometry{verbose}
\usepackage{babel}
\usepackage{extarrows}
\usepackage[colorlinks]{hyperref}
\usepackage{listings}
\usepackage{color,xcolor}
\usepackage{graphicx}
\usepackage{subfigure} 
\usepackage{amsthm,amssymb,amsfonts}
\usepackage{textcomp}
\usepackage{bm}
\usepackage{booktabs}

\providecommand{\U}[1]{\protect\rule{.1in}{.1in}}
\topmargin            -18.0mm
\textheight           226.0mm
\oddsidemargin      -4.0mm
\textwidth            166.0mm
\def\baselinestretch{1.5}

\DeclareMathAlphabet\mathbfcal{OMS}{cmsy}{b}{n}
\newcommand{\Ab}{\mathbf{A}}
\newcommand{\Bb}{\mathbf{B}}
\newcommand{\Ub}{\mathbf{U}}
\newcommand{\Vb}{\mathbf{V}}
\newcommand{\ub}{\mathbf{u}}
\newcommand{\vb}{\mathbf{v}}
\newcommand{\Ucal}{\mathbfcal{U}}
\newcommand{\Wcal}{\mathbfcal{W}}
\newcommand{\Vcal}{\mathbfcal{V}}
\newcommand{\Xcal}{\mathbfcal{X}}
\newcommand{\Ycal}{\mathbfcal{Y}}
\newcommand{\Rbb}{\mathbb{R}}

\begin{document}

\begin{center}
	\textbf{\LARGE{SI231 - Matrix Computations, 2021 Fall}}\\
	{\Large Homework Set \#1}\\
	\texttt{Prof. Yue Qiu}
\par\end{center}

\noindent
\rule{\linewidth}{0.4pt}
% \noindent
% \rule{\linewidth}{0.4pt}
{\bf Acknowledgements:}
\begin{enumerate}
	\item Deadline: {\bf \textcolor{red}{2021-10-12 23:59:59}}
	\item \textbf{Late Policy details} can be found on piazza.
	\item Submit your homework in \textbf{Homework 1} on \textbf{Gradscope}. Entry Code: \textbf{2RY68R}. Make sure that you have correctly select pages for each problem. If not, you probably will get 0 point.
	\item No handwritten homework is accepted. You need to write \LaTeX. (If you have difficulties in using \LaTeX, you are allowed to use \textbf{MS Word or Pages} for the first and the second homework to accommodate yourself.)
	\item Use the given template and give your solution in English. Solution in Chinese is not allowed.
	\item Your homework should be uploaded in the PDF format, and the naming format of the file is not specified.
\end{enumerate}
\rule{\linewidth}{0.4pt}

\section{Vector space and subspace}

\noindent\textbf{Problem 1}. \textcolor{blue}{(6 points $\times$ 3)}
\begin{enumerate}
    \item Let $\Xcal$ and $\Ycal$ be two subspaces of a vector space $\Vcal$:
    \begin{enumerate}
        \item Prove that the intersection $\Xcal\cap\Ycal$ is also a subspace of $\Vcal$.
        \item Show that the union of $\Xcal\cup\Ycal$ need not to be a subspace of $\Vcal$.
    \end{enumerate}
    \item Prove or give a counterexample:
    \begin{enumerate}
        \item If $\Ucal_1$, $\Ucal_2$, and $\Wcal$ are subspaces of $\Vcal$ such that $\Ucal_1+\Wcal = \Ucal_2+\Wcal$, then $\Ucal_1 = \Ucal_2$.
        \item If $\Ucal_1$, $\Ucal_2$, and $\Wcal$ are subspaces of $\Vcal$ such that $\Vcal = \Ucal_1 \oplus \Wcal$ and $\Vcal = \Ucal_2 \oplus \Wcal$, then $\Ucal_1 = \Ucal_2$.\footnote{Let $\mathcal{S}_1$ and $\mathcal{S}_2$ be two subspaces of $\mathbb{R}^n$, if $\mathcal{S}_1 \cap \mathcal{S}_2 = \{ \mathbf{0}\}$ and $\mathcal{S}_1 + \mathcal{S}_2 = \mathbb{R}^n$, we define the \textbf{direct sum} $\mathbb{R}^n = \mathcal{S}_1 \oplus \mathcal{S}_2.$}.
    \end{enumerate}
    \item Let $\Ub = \{\ub_1,\ub_2,...,\ub_r\}$ and $\Vb = \{\ub_1,\ub_2,...,\ub_r,\vb\}$ be two sets of vectors from the same vector space, prove that $span(\Ub) = span(\Vb)$ if and only if $\vb\in span(\Ub)$.
\end{enumerate}

\noindent\textcolor{blue}{
	\textbf{Solution:}
}

\newpage

\section{Basis, dimension and rank}
\noindent\textbf{Problem 1}. \textcolor{blue}{(5 points $\times$ 2)}
For matrix $\mathbf{A}\in\mathbb{R}^{n\times n}$, we have $\mathcal{V} = \{\mathbf{X} \in \mathbb{R}^{n\times n}|\mathbf{AX}=\mathbf{XA}\}$,
\begin{enumerate}
	\item Prove that $\mathcal{V}$ is a linear subspace of the linear space $\mathbb{R}^{n\times n}$;
	\item If $\mathbf{A} = \left( 
  \begin{array}{ccc}  
    1 & 1\\  
    2 & -1 
  \end{array}
\right)  $ , please give a basis and the dimension of $\mathcal{V}$.
\end{enumerate}

\noindent\textcolor{blue}{
	\textbf{Solution:}
}

\newpage
\noindent\textbf{Problem 2}. \textcolor{blue}{(5 points)}
The linear space $\mathcal{S}$ contains the following polynomials: $f_1(t) = 1 + 4t - 2t^2 + t^3 $, $f_2(t) = -1 + 9t - 3t^2 + 2t^3 $, $f_3(t) = -5 + 6t + t^3 $, $f_4(t) = 5 + 7t - 5t^2 + 2t^3 $. Please give the rank of the quadruple $(f_1(t),f_2(t),f_3(t),f_4(t))$ and its maximal linearly independent set.

\noindent\textcolor{blue}{
	\textbf{Solution:}
}

\newpage
\noindent\textbf{Problem 3}. \textcolor{blue}{(5 points $\times$ 2)}
For any matrix $\mathbf{A}\in\mathbb{R}^{n\times n}$, $\mathcal{S}_1 = \{\mathbf{A} \in \mathbb{R}^{n\times n}|\mathbf{A}^T=\mathbf{A}\}$ and $\mathcal{S}_2 = \{\mathbf{A} \in \mathbb{R}^{n\times n}|\mathbf{A}^T=-\mathbf{A}\}$ are two subspaces of $\mathbb{R}^{n\times n}$, 
\begin{enumerate}
	\item Prove that $ \mathbb{R}^{n\times n} =\mathcal{S}_1 \oplus \mathcal{S}_2$.
	\item If n = 3,  please give a basis of $\mathcal{S}_1$ and the dimension of $\mathcal{S}_2$.
\end{enumerate}

\noindent\textcolor{blue}{
	\textbf{Solution:}
}


\newpage
\section{Four fundamental subspaces}

\noindent\textbf{Problem 1}. \textcolor{blue}{(2 points + 5 points)}
For an $n$ $\times$ $m$ real matrix $\mathbf{A}$.
\begin{enumerate} 
\item Determine the relationship of $dim(\mathcal{R}(\mathbf{A}))$, $dim(\mathcal{N}(\mathbf{A}))$, and $rank(\mathbf{A})$. 

\item Prove that $\mathcal{N}(\mathbf{A}) \oplus \mathcal{R}(\mathbf{A}^T)=\mathbb{R}^{m}$.
\end{enumerate}

\noindent\textcolor{blue}{
	\textbf{Solution:}
}


\newpage

\noindent\textbf{Problem 2}. \textcolor{blue}{(3 points + 5 points $\times$ 3)}
Given matrices $\Ab,\Bb\in \Rbb^{n\times m}$ and $rank(\left[ \mathbf{A},\mathbf{B} \right] ) = n$.

\begin{enumerate}
    \item Determine the relationship between $dim(\mathcal{N}(\mathbf{A}^T)) $ and $dim(\mathcal{R}(\mathbf{B})) $.% in general case.
    \item If $ \mathbf{A}^T \mathbf{B}=0 $, determine the relationship between $\mathcal{N}(\mathbf{A}^T) $ and $\mathcal{R}(\mathbf{B}) $. 
     \item Please determine the rank of $\left( 
  \begin{array}{cc}  
    \mathbf{A} & 0\\  
    0 & \mathbf{B}
  \end{array}
  \right) $.
     \item Please determine the $\bf{Supremacy}$ and $\bf{Infimum}$ of the rank of $\left(
  \begin{array}{cc}  
    \mathbf{A}\\  
    \mathbf{B}
  \end{array}
  \right) $ using $m$ or $n$. \\
  (All Matrix $\mathbf{A}$'s, and $\mathbf{B}$'s that satisfy the mentioned condition)  
 \end{enumerate}

\noindent\textcolor{blue}{
	\textbf{Solution:}
}

\newpage
\section{Vector norm and matrix norm}
\noindent\textbf{Problem 1}. \textcolor{blue}{(5 points $\times$ 3)}
 The Frobenius norm of a $\mathbb{R}^{n \times m}$ matrix $\mathbf{A}$ defined as the square root of the sum of the absolute squares of its elements, 
    $$\Vert  \mathbf{A} \Vert_F \equiv \sqrt{\sum_{i =1 }^m \sum_{j=1}^n |a_{ij}|^2}, $$
   it also equal to the square root of the matrix trace of $\mathbf{A}^T\mathbf{A}$, where $\mathbf{A}^T$ is the  transpose of $\mathbf{A}$,
    $$ \Vert  \mathbf{A} \Vert_F = \sqrt{\textbf{Tr}(\mathbf{A}^T\mathbf{A})}. $$
\begin{enumerate}
    \item Show that Frobenius norm is a matrix norm. \\
    \textbf{Hint}: You may use the Cauchy-Schwarz inequality $$\|\Ab\Bb\|_F \leq \|\Ab\|_F\|\Bb\|_F$$
    \item The spectral norm of a matrix $\mathbf{A}$ is the largest singular value of $\mathbf{A}$ (the square root of the largest eigenvalue of the matrix $\mathbf{AA}^T$,
    $$\Vert \mathbf{A} \Vert_2 = \sqrt{\lambda_{max}(\mathbf{AA}^T)} .$$\
    Show that $\Vert \mathbf{A} \Vert_2 \leq \Vert  \mathbf{A} \Vert_F \leq \sqrt{n} \Vert \mathbf{A} \Vert_2$ 
    \item Suppose $\mathbf{A} = \mathbf{xy}^T$, where $\mathbf{x,y} \in \mathbb{R}^n$, show that 
    $$\Vert \mathbf{A} \Vert_F^2 = \Vert \mathbf{x} \Vert_2^2\Vert \mathbf{y} \Vert_2^2$$
\end{enumerate}
\noindent\textcolor{blue}{
	\textbf{Solution:}
}

\newpage
\section{Projector and projection}
\noindent\textbf{Problem 1}. \textcolor{blue}{(2 points+5 points $\times$ 3)}
A rotation matrix $\mathbf{R} \in \mathbb{R}^{n \times n}$ is an orthogonal matrix $(\mathbf{RR}^T = \mathbf{I_n})$.\footnote{$\mathbf{I_n}$ is the identity matrix of size $n\times n$}
\begin{enumerate}
    \item According to the above definition, find all rotation matrices in $\mathbb{R}^{2 \times 2}$.
    \item Let $\mathbf{R}_1$ and $\mathbf{R}_2$ be the rotation matrices in $\mathbb{R}^{2 \times 2}$, if $\mathbf{R}_1$ is rotation through $\alpha_1$ and $\mathbf{R}_2$ is rotation through $\alpha_2$. Consider: is  $\mathbf{R}_1 \mathbf{R}_2$ the rotation matrix. If the answer is "yes", what is the angle of rotation, or else explain why the answer is "no".
    \item For arbitrarily rotation matrix $\mathbf{R} \in \mathbb{R}^{n \times n}$, if $\mathbf{S} = (\mathbf{R}-\mathbf{I_n})(\mathbf{R}+\mathbf{I_n})^{-1}$, show that $\mathbf{S}$ is a skew symmetric matrix ($\mathbf{S}^T = -\mathbf{S}$ )
    \item If $\mathbf{S} \in \mathbb{R}^{n \times n}$ is a skew symmetric matrix, show that $\mathbf{R} = (\mathbf{I_n}-\mathbf{S})^{-1} (\mathbf{I_n} +\mathbf{S})$ is a rotation matrix.
\end{enumerate}
\noindent\textcolor{blue}{
	\textbf{Solution:}
}

\end{document}
